Complex numbers are in the form of \( a + bi \), where \( a \) and \( b \) are real numbers and \( i \) is the imaginary unit.
i.e. \( i = \sqrt{-1} \)

\(i^2 = -1\)

\section{General form and polar form of complex numbers}
The \textbf(polar form of a complex number) is \(r \cos (\theta) + i \sin(\theta)\)

General form \(\mathbb{Z} = a + bi\)

To find the polar form from the general form, use the following formulas:
r = \(\sqrt{a^2 + b^2}\) -
 \(\theta = \arctan(\frac{b}{a})\)

For example, given the complex number \(3 + 4i\), the polar form is:
 r = \(\sqrt{3^2 + 4^2} = 5\) -
\(\theta = \arctan(\frac{4}{3}) = 53.13\)


\section{Complex conjugate}
\section(Argand diagram)

\section{Complex numbers as roots of cubic equations}

\section{Demoivre's theorem}


\section{Proof of Demoivre's theorem}
