% Define a new command for elegant math without background
\newcommand{\elegantmath}[1]{\textcolor{blue!70!black}{#1}}

\chapter{Introduction to Elegant Math Display}

\section{An Example Function}
Here is an example of how you can display a function elegantly, making it a natural part of the page:

\[
\elegantmath{f(x) \text{ is a function defined on the closed interval } [a, b]}
\]

\section{Solving an Equation}
Consider the following equation:

\[
\elegantmath{\lim_{x \to 2} \frac{x^2 - 4}{x - 2}}
\]

The solution involves factoring the numerator:

\[
\elegantmath{\frac{(x - 2)(x + 2)}{x - 2} = x + 2}
\]

As \(x\) approaches 2, the limit is:

\[
\elegantmath{\lim_{x \to 2} (x + 2) = 4}
\]

\section{A More Complex Example}
Let's consider a slightly more involved example:

\[
\elegantmath{\int_{0}^{\pi} \sin(x) \, dx = -\cos(x) \Big|_{0}^{\pi} = -(-1) - (-1) = 2}
\]

\[
\elegantmath{\int_{0}^{\pi} \sin(x) \, dx = -\cos(x) \Big|_{0}^{\pi} = -(-1) - (-1) = 2}
\]
\[
\elegantmath{\int_{0}^{\pi} \sin(x) \, dx = -\cos(x) \Big|_{0}^{\pi} = -(-1) - (-1) = 2}
\]
\[
\elegantmath{\int_{0}^{\pi} \sin(x) \, dx = -\cos(x) \Big|_{0}^{\pi} = -(-1) - (-1) = 2}
\]
\[
\elegantmath{\int_{0}^{\pi} \sin(x) \, dx = -\cos(x) \Big|_{0}^{\pi} = -(-1) - (-1) = 2}
\]
\[
\elegantmath{\int_{0}^{\pi} \sin(x) \, dx = -\cos(x) \Big|_{0}^{\pi} = -(-1) - (-1) = 2}
\]